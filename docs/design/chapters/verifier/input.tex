\section{Input Data Structures}

All input data divided into two parts:

\begin{enumerate}
    \item Placeholder proof byteblob itself;
    \item Verification parameters used to verify proof.
\end{enumerate}

\subsection{Placeholder Proof Structure}

Placeholder proof consists of different fields and some of them are of complex structure types, which will be described in top-down order.

So, the first one Placeholder proof has the following structure, which is described in pseudocode:

\begin{verbatim}
    struct PlaceholderProof {
        witness_commitment: vector<uint8>
        v_perm_commitment: vector<uint8>
        input_perm_commitment: vector<uint8>
        value_perm_commitment: vector<uint8>
        v_l_perm_commitment: vector<uint8>
        T_commitment: vector<uint8>
        challenge: uint256
        lagrange_0: uint256
        witness: LPCProof
        permutation: LPCProof
        quotient: LPCProof
        lookups: vector<LPCProof>
        id_permutation: LPCProof
        sigma_permutation: LPCProof
        public_input: LPCProof
        constant: LPCProof
        selector: LPCProof
        special_selectors: LPCProof
    }
\end{verbatim}

In turn proof of LPC algorithm has the following structure:

\begin{verbatim}
    struct LPCProof {
        T_root: vector<uint8>
        z: vector<vector<uint256>>
        fri_proofs: vector<FRIProof>
    }
\end{verbatim}

The next one description is for structure of FRI algorithm proof:

\begin{verbatim}
    struct FRIProof {
        final_polynomials: vector<vector<uint256>>
        round_proofs: vector<FRIRoundProof>
    }
\end{verbatim}

One of the components of the FRI algorithm proof is so called round FRI proof, which has the following structure:

\begin{verbatim}
    struct FRIRoundProof {
        colinear_value: vector<uint256>
        T_root: vector<uint256>
        colinear_path: MerkleProof
        p: vector<MerkleProof>
    }
\end{verbatim}

The next important component is the merkle tree proof of the following structure:

\begin{verbatim}
    struct MerkleProof {
        leaf_index: uint64
        root: vector<uint8>
        path: vector<MerkleProofLayer>
    }
\end{verbatim}

\begin{verbatim}
    struct MerkleProofLayer {
        layer: vector<MerkleProofLayerElement>
    }       
\end{verbatim}

In the simplest and used case of the merkle tree with arity 2 layer consists of only one element:

\begin{verbatim}
    struct MerkleProofLayerElement {
        position: uint64
        hash: vector<uint8>
    }
\end{verbatim}

It is important to note that before sending Placeholder proof to EVM for verification it should be serialized into byteblob format,
which is done using corresponding marshalling module 
(\url{https://github.com/NilFoundation/crypto3-zk-marshalling/blob/01b531550a99232586e17c1e383e4693a4ddc924/include/nil/crypto3/marshalling/zk/types/placeholder/proof.hpp}).

\subsection{Verification Parameters}

Verification parameters are used to parametrize Placeholder algorithm depending on chosen security parameters and specific circuit for which proof was created.

Following parameters are required to complete Placeholder verification procedure in-EVM:

\begin{verbatim}
    uint256_t modulus; // modulus of chosen prime field
    uint256_t r; // parameter of FRI algorithm
    uint256_t max_degree; // parameter of FRI algorithm
    uint256_t lambda; // parameter of LPC algorithm
    uint256_t rows_amount; // parameter defined by chosen circuit
    uint256_t omega; // parameter defined by chosen circuit
    uint256_t max_leaf_size; // parameter dependent on specific instance of Placeholder algorithm, equal to max leaf size among all of the instances of batched LPC algorithm used within Placeholder algorithm
    std::vector<uint256_t> domains_generators; // parameter defined by chosen circuit 
    std::vector<uint256_t> q_polynomial; // parameter of Placeholder algorithm
    std::vector<std::vector<uint256_t>> columns_rotations; //parameter defined by chosen circuit
\end{verbatim}
