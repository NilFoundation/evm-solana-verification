\section{Input Data Structures}

All input data divided into two parts:

1. Placeholder proof byteblob itself;
2. Verification parameters used to verify proof.

PLACEHOLDER PROOF STRUCTURE

Placeholder proof consists of different fields and some of them are of complex structure types, which will be described in top-down order.

So, the first one Placeholder proof has the following structure, which is described in pseudocode:

struct PlaceholderProof {
    witness_commitment: vector<uint8>
    v_perm_commitment: vector<uint8>
    input_perm_commitment: vector<uint8>
    value_perm_commitment: vector<uint8>
    v_l_perm_commitment: vector<uint8>
    T_commitment: vector<uint8>
    challenge: uint256
    lagrange_0: uint256
    witness: LPCProof
    permutation: LPCProof
    quotient: LPCProof
    lookups: vector<LPCProof>
    id_permutation: LPCProof
    sigma_permutation: LPCProof
    public_input: LPCProof
    constant: LPCProof
    selector: LPCProof
    special_selectors: LPCProof
}

In turn proof of LPC algorithm has the following structure:

struct LPCProof {
    T_root: vector<uint8>
    z: vector<vector<uint256>>
    fri_proofs: vector<FRIProof>
}

The next one description is for structure of FRI algorithm proof:

struct FRIProof {
    final_polynomials: vector<vector<uint256>>
    round_proofs: vector<FRIRoundProof>
}

One of the components of the FRI algorithm proof is so called round FRI proof, which has the following structure:

struct FRIRoundProof {
    colinear_value: vector<uint256>
    T_root: vector<uint256>
    colinear_path: MerkleProof
    p: vector<MerkleProof>
}

The next important component is the merkle tree proof of the following structure:

struct MerkleProof {
    leaf_index: uint64
    root: vector<uint8>
    path: vector<MerkleProofLayer>
}

struct MerkleProofLayer {
    layer: vector<MerkleProofLayerElement>
}

In the simplest and used case of the merkle tree with arity 2 layer consists of only one element:

struct MerkleProofLayerElement {
    position: uint64
    hash: vector<uint8>
}

It is important to note that before sending Placeholder proof to EVM for verification it should be serialized into byteblob format, which is done using corresponding marshalling module (https://github.com/NilFoundation/crypto3-zk-marshalling/blob/01b531550a99232586e17c1e383e4693a4ddc924/include/nil/crypto3/marshalling/zk/types/placeholder/proof.hpp).

VERIFICATION PARAMETERS

Verification parameters are used to parametrize Placeholder algorithm depending on chosen security parameters and specific circuit for which proof was created.

Following parameters are required to complete Placeholder verification procedure in-EVM:

1. modulus: uint256
2. r: uint256
3. max_degree: uint256
4. lambda: uint256
5. rows_amount: uint256
6. omega: uint256
7. max_leaf_size: uint256
8. Domains generators: vector<uint256>
9. q polynomial: vector<uint256>
10. Columns rotations: vector<vector<int256>>
