\section{Input Data Structures}

All input data divided into two parts:

\begin{enumerate}
    \item Placeholder proof byteblob itself;
    \item Verification parameters used to verify proof.
\end{enumerate}

\subsection{Placeholder Proof Structure}

Placeholder proof consists of different fields and some of them are of complex structure types, which will be described in top-down order.

So, the first one Placeholder proof has the following structure, which is described in pseudocode:

\begin{verbatim}
    struct placeholder_proof {
        std::vector<uint8_t> witness_commitment;
        std::vector<uint8_t> v_perm_commitment;
        std::vector<uint8_t> input_perm_commitment;
        std::vector<uint8_t> value_perm_commitment;
        std::vector<uint8_t> v_l_perm_commitment;
        std::vector<uint8_t> T_commitment;
        uint256_t challenge;
        uint256_t lagrange_0;
        lpc_proof witness;
        lpc_proof permutation;
        lpc_proof quotient;
        std::vector<lpc_proof> lookups;
        lpc_proof id_permutation;
        lpc_proof sigma_permutation;
        lpc_proof public_input;
        lpc_proof constant;
        lpc_proof selector;
        lpc_proof special_selectors;
    }
\end{verbatim}

In turn proof of LPC algorithm has the following structure:

\begin{verbatim}
    struct lpc_proof {
        std::vector<uint8_t> T_root;
        std::vector<std::vector<uint256_t>> z;
        std::vector<fri_proof> fri_proofs;
    }
\end{verbatim}

The next one description is for structure of FRI algorithm proof:

\begin{verbatim}
    struct fri_proof {
        std::vector<std::vector<uint256_t>> final_polynomials;
        std::vector<fri_round_proof> round_proofs;
    }
\end{verbatim}

One of the components of the FRI algorithm proof is so called round FRI proof, which has the following structure:

\begin{verbatim}
    struct fri_round_proof {
        std::vector<uint256_t> colinear_value;
        std::vector<uint256_t> T_root;
        merkle_proof colinear_path;
        std::vector<merkle_proof> p;
    }
\end{verbatim}

The next important component is the merkle tree proof of the following structure:

\begin{verbatim}
    struct merkle_proof {
        uint64_t leaf_index;
        std::vector<uint8_t> root;
        std::vector<merkle_proof_layer> path;
    }
\end{verbatim}

\begin{verbatim}
    struct merkle_proof_layer {
        std::vector<merkle_proof_layer_element> layer;
    }       
\end{verbatim}

In the simplest and used case of the merkle tree with arity 2 layer consists of only one element:

\begin{verbatim}
    struct merkle_proof_layer_element {
        uint64_t position;
        std::vector<uint8_t> hash;
    }
\end{verbatim}

It is important to note that before sending Placeholder proof to EVM for verification it should be serialized into byteblob format,
which is done using corresponding marshalling module 
(\url{https://github.com/NilFoundation/crypto3-zk-marshalling/blob/01b531550a99232586e17c1e383e4693a4ddc924/include/nil/crypto3/marshalling/zk/types/placeholder/proof.hpp}).

\subsection{Verification Parameters}

Verification parameters are used to parametrize Placeholder algorithm depending on chosen security parameters and specific circuit for which proof was created.

Following parameters are required to complete Placeholder verification procedure in-EVM:

\begin{itemize}
    \item \texttt{uint256_t modulus} - modulus of chosen prime field
    \item \texttt{uint256_t r} - FRI algorithm rounds
    \item \texttt{uint256_t max\_degree} - maximal degree of polynomials for commitment scheme
    \item \texttt{uint256_t lambda} - parameter of LPC algorithm
    \item \texttt{uint256_t rows\_amount} - parameter defined by chosen circuit
    \item \texttt{uint256_t omega} - parameter defined by chosen circuit
    \item \texttt{uint256_t max\_leaf\_size} - parameter dependent on specific instance of Placeholder algorithm,
        equal to max leaf size among all of the instances of batched LPC algorithm used within Placeholder algorithm
    \item \texttt{std::vector<uint256_t> Domains generators} - parameter defined by chosen circuit 
    \item \texttt{std::vector<uint256_t> q\_polynomial} - FRI folding-related parameter
    \item \texttt{std::vector<std::vector<int256_t>> columns\_rotations} - parameter defined by chosen circuit
\end{itemize}
